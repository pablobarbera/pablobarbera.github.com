\documentclass[margin,line,11pt]{resume}
\usepackage[colorlinks=true, pdfstartview=FitV, linkcolor=blue, urlcolor=blue]{hyperref}
\usepackage{hyperref}

\usepackage{fontawesome,url}
\usepackage{eurosym}
\usepackage{libertine}
\usepackage{enumitem}

\setmonofont[Mapping=tex-text,Scale=0.9]{Helvetica} 

\def\myemail{pbarbera@usc.edu}
\def\myweb{www.pablobarbera.com}
\def\mytwitter{@p\_barbera}

\newcommand{\nl}{\vspace{0.10in}\\}

%______________________________________________________________________________________________________________________
\begin{document}
\name{\Large Pablo Barber\'{a}}
\begin{resume}

    %__________________________________________________________________________________________________________________
    % Contact Information
    \section{\mysidestyle Contact Information}

    \begin{tabular*}{\textwidth}{ l @{\extracolsep{\fill}} r}
    School of International Relations    		& \texttt{\href{mailto:\myemail}{\myemail}} \, \faEnvelope \\
    University of Southern California 			& \texttt{\href{\myweb}{\myweb}} \, \faGlobe \\
    3518 Trousdale Parkway, VKC 330		& \texttt{\href{http://twitter.com/p_barbera}{\mytwitter}} \, \faTwitter \\
    Los Angeles, CA 90089			&  \texttt{\href{http://www.github.com/pablobarbera}{github.com/pablobarbera}} \, \faGithub \\
    \end{tabular*}

%__________________________________________________________________________________________________________________

    \section{\mysidestyle Academic\\Positions}
    \textbf{London School of Economics and Political Science}
    \begin{itemize}
    \item[] Assistant Professor of Computational Social Science, Methodology Department, 2018--
    \end{itemize}
    \vspace{-.30cm}
    \textbf{University of Southern California}
    \begin{itemize}
    \item[] Assistant Professor, School of International Relations, 2016--2017
    \vspace{-.15cm}
    \item[] Principal Investigator, Networked Democracy Lab (\textit{netdem.org})
    \end{itemize}
    \vspace{-.30cm}
    \textbf{New York University}
    \begin{itemize}
    \item[] Moore-Sloan Fellow, Center for Data Science, 2015--2016
    \end{itemize}
        
   %__________________________________________________________________________________________________________________
    % Education
    \section{\mysidestyle Education}
\textbf{New York University}
\begin{itemize}
\item[] Ph.D. in Political Science, May 2015
\vspace{-.15cm}
\item[] $\ast$ Committee: Jonathan Nagler, Joshua A. Tucker, Patrick Egan, Neal Beck, Richard Bonneau
\vspace{-.15cm}
\item[] $\ast$  Lab: Social Media and Political Participation (SMaPP) laboratory
\end{itemize}
\vspace{-.30cm}
\textbf{European University Institute} 
\begin{itemize}
\item[] Master of Research in Political Science, 2010
\end{itemize}
\vspace{-.30cm}
\textbf{Pompeu Fabra University} 
\begin{itemize}
\item[] M.A. in Political and Social Sciences, 2009
\vspace{-.15cm}
\item[] B.A. in Political Science and Administration (with highest honors), 2008 
\end{itemize}

%__________________________________________________________________________________________________________________
    % Research Interests
    \section{\mysidestyle Research\\Interests}
\emph{Methodological}: Computational Social Science, Social Network Analysis, Text as Data Methods,  Bayesian Data Analysis, Ideal Point Estimation.
\vspace{-.30cm}

\emph{Substantive}: Social Media and Elections, Comparative Electoral Behavior, Political Corruption and Accountability.
    
    
    %__________________________________________________________________________________________________________________
    % Publications
    \section{\mysidestyle Peer-Reviewed\\Publications}
\vspace{.15cm}    
\begin{itemize}[leftmargin=5.5mm]
\item[16.] Jost, J. T., \textbf{Barber\'{a}, P.}, Bonneau, R., Langer, M., Metzger, M., Nagler, J., Sterling, J. \& Tucker, J. A. \href{http://onlinelibrary.wiley.com/doi/10.1111/pops.12478/full}{``How Social Media Facilitates Political Protest: Information, Motivation, and Social Networks''}, \textit{Political Psychology}, 2018, 39(S1), 85-118.

\item[15.] \textbf{Barber\'{a}, P.} \& Zeitzoff, T. \href{https://academic.oup.com/isq/article/doi/10.1093/isq/sqx047/4430887/The-New-Public-Address-System-Why-Do-World-Leaders}{``The New Public Address System: Why Do World Leaders Adopt Social Media?''}, \textit{International Studies Quarterly}, 2017, forthcoming.
    
\item[14.] Bauer, P., \textbf{Barber\'{a}, P.}, Ackerman, K. \& Venetz, A. ``\href{http://link.springer.com/article/10.1007/s11109-016-9368-2}{Is the left-right scale a valid measure of ideology? Investigating the meaning of `left' and `right' and its impact on measurement}'', \textit{Political Behavior}, 2017, 39 (3): 553-583.

\item[13.] Riera, P., G\'{o}mez, R., Mayoral, J.A. \textbf{Barber\'{a}, P.} \& Montero, J.R. ``\href{http://revintsociologia.revistas.csic.es/index.php/revintsociologia/article/view/675/832}{Local Elections in Spain: The Personalization of Voting Behavior}'', \textit{Revista Internacional de Sociolog\'{i}a}, 2017, 75 (2): e062.
    
\item[12.] Theocharis, Y., \textbf{Barber\'{a}, P.}, Popa, S., Fazekas, Z. \& Parnet, O. ``\href{http://onlinelibrary.wiley.com/doi/10.1111/jcom.12259/abstract}{A Bad Workman Blames His Tweets. The Consequences of Citizens' Uncivil Twitter Use when Interacting with Party Candidates}'', \textit{Journal of Communication}, 2016, 66 (6): 1007-1031.
\begin{itemize}
\item[] $\ast$ 2016 \href{http://www.apsanet.org/section-18-best-conference-paper-award}{Best Information Technology \& Politics Conference Paper Award}
\end{itemize}
    
\item[11.] Vaccari, C., Valeriani, A., \textbf{Barber\'{a}, P.}, Bonneau, R., Jost, J., Nagler, J., Tucker, J. ``\href{http://sms.sagepub.com/content/2/3/2056305116664221.abstract}{Of echo chambers and contrarian clubs: Exposure to political disagreement among German and Italian users of Twitter}'', \textit{Social Media + Society}, 2016, 2 (3)

\newpage

\item[10.] Riera, P., G\'{o}mez, R., \textbf{Barber\'{a}, P.}, Mayoral, J.A. \& Montero, J.R. ``\href{http://www.cepc.gob.es/publicaciones/revistas/revistaselectronicas?IDR=3&IDN=1360&IDA=37721}{Local Elections in Spain: A Multilevel Analysis of the Individual and Contextual Determinants of Voting}'', \textit{Revista de Estudios Pol\'{i}ticos}, 2016, 172, 47--82.

\item[9.] Fern\'{a}ndez-V\'{a}zquez, P., \textbf{Barber\'{a}, P.} \& Rivero, G. \href{http://dx.doi.org/10.1017/psrm.2015.8}{``Rooting out corruption or rooting for corruption? The Heterogenous Electoral Consequences of Scandals''}, \textit{Political Science Research and Methods}, 2016, 4 (2), 379--397.

\item[8.] \textbf{Barber\'{a}, P.}, Jost, J., Nagler, J., Tucker, J. \& Bonneau, R. ``\href{http://pss.sagepub.com/content/early/2015/08/21/0956797615594620}{Tweeting from Left to Right: Is Online Political Communication More Than an Echo Chamber?}'', \textit{Psychological Science}, 2015, 26 (10), 1531--1542.

\item[7.] \textbf{Barber\'{a}, P.} \href{http://pan.oxfordjournals.org/cgi/reprint/mpu011?ijkey=uMFPw4dsMHM7608&keytype=ref}{``Birds of the Same Feather Tweet Together. Bayesian Ideal Point Estimation Using Twitter Data''}. \emph{Political Analysis}, 2015, 23 (1), 76--91.
\begin{itemize}
\item[] $\ast$ 2015 \href{http://oxfordjournals.org/our_journals/polana/editorschoice.html}{Editors' Choice article Award} for an article published in \textit{Political Analysis} that represents a ``especially significant contribution to political methodology.''
\vspace{-.1cm}
\item[] $\ast$ 2016 \href{http://www.apsanet.org/section-18-best-information-technology-and-politicsp-award}{Best Information Technology \& Politics Article Award}, given for the best published article about information technology and politics in 2015.
\vspace{-.1cm}
\item[] $\ast$ 2016 \href{http://www.oxfordjournals.org/our_journals/polana/awards_warrenmiller.html}{Warren Miller Article Award}, given by the Society of Political Methodology for the best article published in Political Analysis in 2015.
\vspace{-.1cm}
\item[] $\ast$ Featured in \textit{Political Analysis} virtual issue on \href{https://www.cambridge.org/core/journals/political-analysis/issue/online-research-methods/FF4E7D31B561F7FC794AF5894A53E9F3}{Innovative Methods in Online Research}
\end{itemize}


\item[6.] \textbf{Barber\'{a}, P.}, Wang, N., Bonneau, R., Jost, J., Nagler, J., Tucker, J. \& Gonz\'{a}lez-Bail\'{o}n, S. ``\href{http://journals.plos.org/plosone/article?id=10.1371/journal.pone.0143611}{The Critical Periphery in the Growth of Social Protests}", \textit{PLOS ONE}, 2015, 10 (11).

\item[5.] Vaccari, C., Valeriani, A., \textbf{Barber\'{a}, P.}, Bonneau, R., Jost, J., Nagler, J., Tucker, J. \href{http://onlinelibrary.wiley.com/doi/10.1111/jcc4.12108/abstract#.VLRLl9mdVFM.twitter}{``Political Expression and Action on Social Media: Exploring the Relationship Between Lower- and Higher-Threshold Political Activities Among Twitter Users in Italy''}, \emph{Journal of Computer-Mediated Communication}, 2015, 20 (2), 221--239.


\item[4.] \textbf{Barber\'{a}, P.} \& Rivero, G. \href{http://ssc.sagepub.com/content/33/6/712}{``Understanding the political representativeness of Twitter users''}, \textit{Social Science Computer Review}, 2015, 33 (6), 721-729.

\item[3.] Vaccari, C., Valeriani, A., \textbf{Barber\'{a}, P.}, Bonneau, R., Jost, J., Nagler, J., Tucker, J. \href{http://www.rivisteweb.it/doi/10.1426/75245}{``Social Media and Political Communication. A survey of Twitter users during the 2013 Italian general election''}, \emph{Italian Political Science Review}, 2013, 43 (3), 381--410.

\item[2.] Riera, P., \textbf{Barber\'{a}, P.}, G\'{o}mez, R., Mayoral, J.A. \& Montero, J.R \href{http://link.springer.com/article/10.1007/s10611-013-9479-1}{``The Electoral Consequences of Corruption Scandals in Spain''}, \emph{Crime, Law and Social Change}, 2013, 60 (5), 515--534.

\item[1.] \textbf{Barber\'{a}, P} \href{http://www.reis.cis.es/REIS/PDF/REIS_132_021285919804928.pdf}{``Voting for Parties or for Candidates? The Trade-Off Between Party and Personal Representation in Spanish Regional and Local Elections''}, \emph{Revista Espa\~{n}ola de Investigaciones Sociol\'{o}gicas}, 2010

\end{itemize}    

%__________________________________________________________________________________________________________________

\section{\mysidestyle Other\\Publications}
    
    \vspace{.15cm}    
\begin{itemize}[leftmargin=5.5mm]
\item[1.]  Tucker, J., Theocharis, Y., Roberts, M. \& \textbf{Barber\'{a}, P.} \href{https://muse.jhu.edu/article/671987/pdf}{``From Liberation to Turmoil: Social Media and Democracy''}, \textit{Journal of Democracy}, 2017, 28 (4), 46-59.   
\end{itemize}        %__________________________________________________________________________________________________________________

    \section{\mysidestyle Book\\Chapters} 
\vspace{.15cm}    
\begin{itemize}[leftmargin=5.5mm]
\item[4.] Klasnja, M., \textbf{Barber\'{a}, P.}, Beauchamp, N., Nagler, J. \& Tucker, J. \href{http://www.oxfordhandbooks.com/view/10.1093/oxfordhb/9780190213299.001.0001/oxfordhb-9780190213299-e-3}{``Measuring Public Opinion with Social Media Data''.} 2017. \textit{The Oxford Handbook of Polling and Polling Methods}, edited by M. Alvarez. Oxford University Press.

\item[3.] \textbf{Barber\'{a}, P.}, Vaccari, A. \& Valeriani, A. \href{https://link.springer.com/chapter/10.1057/978-1-137-59890-5_2}{``Social Media, Personalisation of News Reporting, and Media Systems' Polarisation in Europe''}. 2017. \textit{Social media and European politics: Rethinking power and legitimacy in the digital era}, edited by M. Barisione and A. Michailidou. Palgrave Macmillan.
    
\item[2.] \textbf{Barber\'{a}, P.}, \href{https://www.practicereproducibleresearch.org/case-studies/barbera.html}{``The Trade-Off Between Reproducibility and Privacy in the Use of Social Media Data to Study Political Behavior''} 2017. \textit{The Practice of Reproducible Research}, edited by Justin Kitzes. University of California Press (forthcoming).

\newpage
    
\item[1.] Tucker, J., Nagler, J., Metzger, M., \textbf{Barber\'{a}, P.}, Penfold-Brown, D. \& Bonneau, R. ``Big Data, Social Media, and Protest: Foundations for a Research Agenda''. 2016. \textit{Computational Social Science. Discovery and Prediction}, edited by R. Michael Alvarez. Cambridge University Press.
\end{itemize}
%    \section{\mysidestyle Under\\Review} 

    % Work in progress
    \section{\mysidestyle Working\\Papers} 
    
\href{http://www.pablobarbera.com/static/barbera_polarization_APSA.pdf}{``How Social Media Reduces Mass Political Polarization. Evidence from Germany, Spain, and the United States.''} 
\begin{itemize}
\item[] $\ast$ 2016 \href{http://www.apsanet.org/PROGRAMS/APSA-Awards/Franklin-L-Burdette-Pi-Sigma-Alpha-Award}{Franklin L. Burdette/Pi Sigma Alpha Award} for Best Paper Presented at APSA 2015.
\end{itemize}

\href{http://pablobarbera.com/static/less-is-more.pdf}{``Less is More? How Demographic Sample Weights can Improve Public Opinion Estimates Based on Twitter Data.''}

\href{http://pablobarbera.com/static/barbera_twitter_responsiveness.pdf}{``Leaders or Followers? Measuring Political Responsiveness in the U.S. Congress Using Social Media Data''} (with Andreu Casas, Richard Bonneau, Patrick Egan, John T. Jost, Jonathan Nagler, and Joshua Tucker).

\href{http://politics.as.nyu.edu/docs/IO/2798/econmedia_mpsa2016_methods_rvBBLMN.pdf}{``Methodological Challenges in Estimating Tone: Application to News Coverage of the U.S. Economy''} (with Amber Boydstun, Suzanne Linn, Jonathan Nagler, and Ryan McMahon)

%``Tracking the spread of misinformation on Twitter. Evidence from the 2016 U.S. election'' (with David Garc\'{i}a)

``From riot police to tweets: How world leaders use social media during contentious politics.'' (with Anita Gohdes, Thomas Zeitzoff, and Evgeniia Iakhnis)

``Tweeting to the choir? Measuring exposure to ideologically diverse campaign messages on Twitter.'' (with Joshua Timm and Javier Lorenzo)



%__________________________________________________________________________________________________________________
    % Grants

        \section{\mysidestyle Grants}
        
\textit{How online incivility affects the interactions between public officials and citizens on Facebook,} \$35,000. Facebook Research Faculty Grant. 2018.        
        
\textit{Paying Attention to Attention: Media Exposure and Opinion Formation in an Age of Information Overload,} 736,600\euro. Volkswagen Foundation. Co-principal investigator, with Simon Munzert, Andrew Guess, JungHwan Yang. 2017.

    \textit{Electoral Behavior in Spanish Local Elections: A Multilevel Approach,} 10,000\euro. Spanish Center for Sociological Research (CIS). Co-principal investigator, with Jos\'{e} Ram\'{o}n Montero, Pedro Riera, Ra\'{u}l G\'{o}mez, and Juan A. Mayoral. 2011.    

  
  %__________________________________________________________________________________________________________________
    % Grants

        \section{\mysidestyle Internal\\Grants}

Center for International Studies, University of Southern California:
\begin{itemize}
\item[] Faculty Research Grant, \$5,000. 2016
\vspace{-.15cm}
\item[] Research Assistantship Grant, \$22,000. 2017
\vspace{-.15cm}
\item[] Conference support, \$15,000. 2017
\end{itemize}
Dornsife College of Letters, Arts and Sciences, University of Southern California:
\begin{itemize}
\item[] Student Opportunities for Academic Research Program, \$3,000. 2016--2017.
\vspace{-.15cm}
\item[] Undergraduate Research Associates Program, \$4,500. 2017--2018.
\end{itemize}
 
    %__________________________________________________________________________________________________________________
    % Fellowships and Awards

    \section{\mysidestyle Fellowships\\ and Awards}
	Warren Miller Prize, Society for Political Methodology, 2016.\nl
    Franklin L. Burdette/Pi Sigma Alpha Award, American Political Science Association, 2016.\nl
	``La Caixa'' Fellowship for Graduate Studies in the U.S., 2010-2012. \nl
	``Salvador de Madariaga'' Fellowship, European University Institute, 2009-2010. \nl
	 2nd National Prize to Excellence in Academic Performance, 2008. \nl
	Pompeu Fabra University graduation prize for Exceptional Achievement, 2008.


\newpage     

            \section{\mysidestyle Teaching}

\emph{Instructor}
\begin{itemize}
\item Fall 2016: European Integration (IR 468)
\item Fall 2016: Introduction to Data Analysis (IR 312)
\item Spring 2017: Introduction to Regression Analysis (POIR 611)
\item Fall 2017: International Relations -- Approaches to Research (IR 211)
\item Fall 2017: Topics in Quantitative Analysis -- \href{http://pablobarbera.com/POIR613/}{Computational Social Science} (POIR 613)
\item Lent 2018: Quantitative Text Analysis (MY 459, co-taught with Ken Benoit)
\end{itemize}


\emph{Workshops \& Summer Schools}
\begin{itemize}
\item \href{https://github.com/pablobarbera/workshop}{Scraping Twitter \& Web Data} and \href{https://github.com/pablobarbera/Rdataviz}{Data Visualization with ggplot2} (4 hours) NYU Politics
\item \href{http://pablobarbera.com/social-media-workshop/}{Collecting and Analyzing Social Media Data with R} (2--4 hours)
\begin{itemize}
\vspace{-.1cm}
\item Georgetown University, 2015
\vspace{-.1cm}
\item University of Cologne, 2017
\vspace{-.1cm}
\item London School of Economics, 2018
\end{itemize}
\item \href{http://pablobarbera.com/SQL-workshop/}{Querying large-scale online datasets: SQL and Google BigQuery} (4 hours). London School of Economics, 2018.
\item \href{https://github.com/pablobarbera/data-science-workshop}{Data Science and Social Science} (18 hours) Center for the Promotion of Research Involving Innovative Statistical Methodology (PRIISM), New York University. (Joint with Alex Hanna and Dan Cervone).
\item \href{https://github.com/pablobarbera/eui-text-workshop}{Automated Text Analysis with R} (6 hours) Quantitative Methods Working Group, European University Institute.
\item \href{https://eventum.upf.edu/event_detail/7273/sections/5650/1st-week-courses.html}{Social Media and Big Data Research} (12 hours) Barcelona Summer School in Survey Methodology, Pompeu Fabra University.
\item \href{https://ecpr.eu/Events/PanelDetails.aspx?PanelID=7055&EventID=116}{Automated Collection of Web and Social Data} (15 hours) ECPR Summer School, Central European University.
\item \href{https://ecpr.eu/Events/PanelDetails.aspx?PanelID=7061&EventID=116}{Big Data Analysis in the Social Sciences} (15 hours) ECPR Summer School, Central European University.
\end{itemize}

\emph{Project Advisor}
\begin{itemize}
\item Fall 2015: \href{http://cds.nyu.edu/academics/ms-in-data-science/curriculum/required-courses/#ds-ga-1006}{Capstone Project in Data Science}, MS in Data Science
\end{itemize}


\emph{Teaching Assistant}
\begin{itemize}
\item January 2014 \& 2015: \href{https://github.com/pablobarbera/NYU-AD-160J}{Social Media and Political Participation} (undergraduate), Joshua Tucker.
\item Fall 2013: \href{https://github.com/pablobarbera/quant3materials}{Quantitative Methods III} (PhD), Neal Beck
\item Fall 2012, Spring 2013: IR Senior Honors Seminar (undergraduate), Michael Gilligan
\end{itemize}

%__________________________________________________________________________________________________________________
    % Invited talks
    
        \section{\mysidestyle Invited\\talks}
2018: Social Science Foo Camp (O'Reilly and Facebook).
        
2017: Second Symposium on Computational Social Science (Chair of Systems Design, ETH Z\"{u}rich), International Workshop on Social Media and Political Participation (SMaPP Global, NYU Abu Dhabi), Center for Public Diplomacy (University of Southern California), Workshop: Navigating bots, algorithms, echo chambers, and disinformation (U.S. State Department \& Stanford University), Seminario sobre Pol\'{i}tica, Redes Sociales y Big Data (Universidad Central de Chile), Department of Methodology (London School of Economics), Methods Speaker Series (University of California San Diego), Comparative Political Behavior Lecture Series (Humboldt University), Workshop: The Future of Humans as Sensors for Social Science Research (UCLA),  International Workshop on Social Media and Political Participation (SMaPP Global, NYU), Research Workshop in American Politics (UC Berkeley), Academic Workshop on Social Media and Political Polarization (Facebook \& Stanford University),  Lecture Series -- Microfoundations of Politics (University of Cologne).

\newpage

2016: Center for Statistics and the Social Sciences (University of Washington), Department of Political Science (University of Copenhagen), International Political Communication Workshop (Universit\'{e} Laval), International Workshop on Social Media and Political Participation (SMaPP Global), Social Media, Methods, and Politics Seminar (Universit\`{a} degli studi di Milano), Text Analysis Conference (University of Amsterdam), AI Seminar (Information Sciences Institute, University of Southern California).

2015: Conference on Political Polarization: Media and Communication Influences (Princeton University, CSDP), Big Data and Political Science Conference (University of Mannheim), Digital Media, Networks, and Political Communication research group (Annenberg School of Communication, UPenn), Workshop on Online Politics (University College Dublin), Microsoft Research (MSR--NY), International Workshop on Social Media and Political Participation (SMaPP Global)

2014: Text as Data Workshop (Northwestern University), Department of Political Science (University of Rochester), Department of Political Science (Penn State), Department of Government (London School of Economics), Institute for Political Economy and Governance (Pompeu Fabra University), School of International Relations (University of Southern California).

2013: Text as Data Workshop (London School of Economics), QMSS Master's Program Speaker Series (Columbia University), Colloquium in Political Behavior (European University Institute)

   
%__________________________________________________________________________________________________________________
    %Other work experience
%                \section{\mysidestyle Other work\\experience}
%    Research Assistant to Professor Jonathan Nagler (NYU), 2012, 2014, 2015\nl
%    Research Assistant, Social Media and Political Participation (SMaPP) Lab, 2013--2015
    
%__________________________________________________________________________________________________________________
    %Skills
%            \section{\mysidestyle Skills}
    
%\emph{Software}

%\begin{itemize}
%\item Advanced: R, JAGS, stan, Stata, \LaTeX 
%\item Intermediate: Python, SPSS, mySQL, Gephi, Netlogo, mongoDB
%\item Code repository: \href{http://www.github.com/pablobarbera}{github.com/pablobarbera}
%\end{itemize}\vspace{-.5cm}
%\emph{Languages}\nl
%\indent Spanish (native), Catalan (fluent), English (fluent)
    
   
%__________________________________________________________________________________________________________________
    %Teaching
    




%__________________________________________________________________________________________________________________
    % Software
    
        \section{\mysidestyle Software}
    
\href{http://cran.r-project.org/web/packages/streamR/}{streamR package}: Access to Twitter Streaming API via R. Available on CRAN. \nl
\href{http://cran.r-project.org/web/packages/Rfacebook/index.html}{Rfacebook package}: Access to Facebook API via R. Available on CRAN. \nl
\href{http://cran.r-project.org/web/packages/instaR/index.html}{instaR package}: Access to Instagram API via R. Available on CRAN. \nl
\href{https://github.com/netdem-USC/netdemR}{netdemR package}: R tools for analysis of Twitter data. Internal R package used at the Networked Democracy Lab at USC. 

%__________________________________________________________________________________________________________________
    %Other work experience
                \section{\mysidestyle Service and\\membership}
\emph{Organizer}: Workshop on Media Exposure and Opinion Formation (USC, Fall 2017).\nl
\emph{Editor}: SAGE Open, special issue on "Social Media and Politics".\nl
\emph{Referee}: \textbf{(see also \href{https://publons.com/author/1291880/pablo-barbera}{Publons profile})} American Political Science Review (x2), American Journal of Political Science (x2), The Journal of Politics (x3), Political Analysis (x3), British Journal of Political Science, Party Politics, Political Studies (x2), Political Science Research and Methods (x2), Research \& Politics (x2) Policy and Internet, ICWSM, Public Opinion Quarterly (x2), Political Communication (x7), Political Research Quarterly, New Media \& Society, Social Science Computer Review (x4), Social Networks, Social Media + Society, Economics \& Politics, American Journal of Sociology, Psychological Science, Journal of Elections, Public Opinion \& Parties (x2); Journal of Communication, Regional \& Federal Studies, Journal of Law and Courts, Kyklos, Wiley, Russell Sage Foundation, South European Society and Politics, Revista Espa\~{n}ola de Investigaciones Sociol\'{o}gicas, Revista Espa\~{n}ola de Ciencia Pol\'{i}tica, MIS Quarterly, Information, Communication, and Society, Communication Methods and Measures, International Journal of Communication.\nl
\emph{Member:} American Political Science Association (Political Methodology, Political Networks, and Information Technology \& Politics sections), European Political Science Association, Midwest Political Science Association.\nl
\emph{Conferences}: chair (EPSA 2013, 2014, 2016, 2017; MPSA 2016, 2017; APSA 2017), discussant (EPSA 2014, 2015; APSA 2015, 2017; MPSA 2016, 2017).\nl
\emph{Member of dissertation committees:} Adam Badawy, Whitney Hua, Fridolin Linder (Penn State)  

\newpage

\emph{Service in Conference Program Committees}: European Political Science Association Conference (2014), SocInfo: Social Informatics (2016), ICWSM: International Conference on Weblogs and Social Media (2017), IC2S2: International Conference on Computational Social Science (2017), General Online Research Conference (2018),  International World Wide Web Conference
-- WWW (2018).\nl
\emph{Service in Award Committees}: chair of APSA Political Networks section Political Ties award committee (2016), chair of APSA Information Technology and Politics section best paper award committee (2017), member of Miller Prize Award committee (2017).


               %__________________________________________________________________________________________________________________
    %Media
                \section{\mysidestyle Writings in Popular Media}
``\href{https://www.washingtonpost.com/news/monkey-cage/wp/2017/12/06/this-explains-how-social-media-can-both-weaken-and-strengthen-democracy/?utm_term=.d52c930452e9}{This explains how social media can both weaken -- and strengthen -- democracy},'' with Joshua Tucker, Yannis Theocharis, and Margaret E. Roberts. Published in The Monkey Cage / Washington Post, 2017.\nl               
``\href{https://www.washingtonpost.com/news/monkey-cage/wp/2016/11/04/twitter-trolls-hurt-democracy-more-than-you-realize-heres-how/}{Twitter trolls are actually hurting democracy},'' with Yannis Theocharis, Zoltan Fazekas, and Sebastian Popa. Published in The Monkey Cage / Washington Post, 2016.\nl
``\href{https://www.washingtonpost.com/news/monkey-cage/wp/2015/11/30/why-everyone-in-a-network-is-important-for-movements-even-the-slactivists/}{Why everyone in a network is important for movements -- even the Slacktivists!},'' with Sandra Gonz\'{a}lez-Bail\'{o}n. Published in The Monkey Cage / Washington Post, 2015.\nl
``\href{http://www.washingtonpost.com/blogs/monkey-cage/wp/2015/06/16/who-is-the-most-conservative-republican-candidate-for-president/}{Who is the most conservative Republican candidate for president?}.'' Published in The Monkey Cage / Washington Post, 2015.\nl
``\href{http://blogs.lse.ac.uk/politicsandpolicy/political-discussions-on-twitter-during-elections-are-dominated-by-those-with-extreme-views/}{Political discussions on Twitter during elections are dominated by those with extreme views},'' with Gonzalo Rivero. European Politics and Policy (EUROPP) Blog, London School of Economics.\nl
 ``\href{http://www.huffingtonpost.com/pablo-barbera/tweeting-the-revolution-s_b_4831104.html}{Tweeting the Revolution: Social Media Use and the \#Euromaidan Protests},'' with Megan Metzger. Published on The Huffington Post, 2014.\nl
``\href{http://www.washingtonpost.com/blogs/monkey-cage/wp/2013/12/04/strategic-use-of-facebook-and-twitter-in-ukrainian-protests/}{How Ukrainian protestors are using Twitter and Facebook},'' with Megan Metzger. Published in The Monkey Cage / Washington Post, 2014.\nl
 ``\href{http://themonkeycage.org/2013/06/01/a-breakout-role-for-twitter-extensive-use-of-social-media-in-the-absence-of-traditional-media-by-turks-in-turkish-in-taksim-square-protests/}{A Breakout Role for Twitter? The Role of Social Media in the Turkish Protests},'' with Megan Metzger. Published in The Monkey Cage / Washington Post, 2013.\nl
 ``\href{http://themonkeycage.org/2013/06/30822/}{The Dynamics of Information Diffusion in the Turkish Protests},'' with Sandra Gonz\'{a}lez-Bail\'{o}n. Published in The Monkey Cage / Washington Post, 2013.\nl
\href{http://www.tuitometro.es}{tuitometro.es}, Real-time tracking of the Spanish 2011 legislative election in Twitter using automatized sentiment analysis, with Gonzalo Rivero (NYU). Featured in the major Spanish newspapers.
  
                \section{\mysidestyle Mentions in Media Outlets}
Vox, ``\href{http://www.vox.com/the-big-idea/2016/12/28/14095452/fake-news-political-bubbles-democracy-facebook}{Fake news aside, Facebook can help puncture our political bubbles},'' Dec 28, 2016.\nl
El Pa\'{i}s, ``\href{http://internacional.elpais.com/internacional/2016/11/25/actualidad/1480095728_565471.html}{As\'{i} influye Facebook en tus opiniones. Las noticias falsas circulan m\'{a}s r\'{a}pido en c\'{a}maras de eco},'' Nov 27, 2016.\nl
Los Angeles Times, ``\href{http://www.latimes.com/politics/la-na-pol-trump-twitter-20161122-story.html}{Trump's Twitterfests are meant to further the culture wars that helped win him the presidency},'' Nov 22, 2016.\nl 
The Economist, ``\href{http://www.economist.com/news/special-report/21695192-social-media-now-play-key-role-collective-action-new-kind-weather}{A new kind of weather},'' Mar 26, 2016.\nl
WIRED, ``\href{http://www.wired.com/2015/12/social-media-is-making-the-debate-on-guns-and-trump-worse/}{Social media is making the debate on guns -- and Trump -- worse},'' Dec 16, 2015.\nl
Quartz, ``\href{http://qz.com/570009/slacktivism-is-having-a-powerful-real-world-impact-new-research-shows/}{Slacktivism is having a powerful real-world impact, new research shows},'' Dec 10, 2015\nl
Vox, ``\href{http://www.vox.com/2015/12/8/9873822/social-media-activism-science}{Changing your Facebook profile picture is doing more good than you might think},'' Dec 8, 2015\nl

\newpage

Pacific Standard, ``\href{https://psmag.com/the-upside-of-slacktivism-2a93294941b0#.hp2xyepdf}{The Upside of Slacktivism},'' Dec 3, 2015\nl
New York Times, ``\href{http://www.nytimes.com/2014/11/21/upshot/social-media-deepens-partisan-divides-but-not-always.html?_r=0}{Social Media Deepens Partisan Divides. But Not Always.},'' Nov 20, 2014.\nl
New York Times, ``\href{http://www.nytimes.com/2014/10/25/upshot/americans-dont-live-in-information-cocoons.html?_r=0}{Americans Don't Live in Information Cocoons},'' Oct 24, 2014\nl
Wall Street Journal, ``\href{http://blogs.wsj.com/washwire/2014/10/21/study-social-media-can-moderate-users-politics/}{Study: Social Media Can Moderate Users' Politics},'' Oct 21, 2014\nl
Nieman Lab, ``\href{http://www.niemanlab.org/2014/10/this-study-finds-that-social-media-use-reduces-political-polarization-instead-of-increasing-it/}{This study finds that social media use reduces political polarization instead of increasing it},'' October 20, 2014\nl
The Atlantic, ``\href{http://www.theatlantic.com/international/archive/2013/06/these-charts-show-how-crucial-twitter-is-for-the-turkey-protesters/276798/}{These Charts Show How Crucial Twitter Is for the Turkey Protesters},'' Jun 12, 2013.\nl
Foreign Policy, ``\href{http://foreignpolicy.com/2013/06/03/is-facebook-a-town-square-or-a-shopping-mall/}{Is Facebook a town square or a shopping mall?},'' June 3, 2014.\nl

%\vspace{1cm}
\centering \textbf{Last updated: \today}
    


\end{resume}
\end{document}
%______________________________________________________________________________________________________________________
% EOF